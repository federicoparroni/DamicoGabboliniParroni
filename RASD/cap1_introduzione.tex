%%CAPITOLO 1: Introduzione=======================================
\chapter{Introduction}

\section{Purpose}

Our team will develop Travlendar+, a calendar-based application that aims to provide a schedule of user appointments, giving a plan to organize his daily life.
The main goals the app must fulfill are:

\begin{enumerate}
\renewcommand\labelenumi{\textbf{G\theenumi}}
\item Schedule user appointments according to his necessities and his preferences, identifying the best mobility options \label{goal:G1}
\item Schedule user appointments according to his necessities and his preferences, identifying the best mobility options \label{goal:G2}
\item Make sure that the user can be in time for his appointments \label{goal:G3}
\item Optimize the schedule with respect to some criteria and constraints chosen by the user \label{goal:G4}
\item Provide a way to move between appointments location using several kinds of travel means \label{goal:G5}
\item Localize public travel means or sharing services, and buy tickets or book a ride, respectively \label{goal:G6}
\item Arrange the trips of the user, allowing him to locate travel services and buy public travel means tickets or book a sharing service \label{goal:G7}
\item Create a system with a \textbf{graphic} user interface, in order to simplify input/output interactions with the user \label{goal:G8}
\end{enumerate}

\section{Scope}
%The world and the machine is a way of thinking about problems and phenomena relevant in the reality of the software system we are to develop. The world contains the phenomena that occours in reality but are not observable by our system; the machine instead contains the phenomena which take place inside the system and are not observable from outside of it. The intersection between these two sets of phenomena is the so called set of shared phenomena, which contains the phenomena that are controlled by the world and observed by the machine or controlled by the machine and observed by the world.
Here we provide a brief description of the aspects of the reality of interest which the application is going to interact with.

User can receive an appointment on a certain date, time and location (over a region), that can be reached using different available travel means. The appointment can be held either at a specific time or in a time interval and lasts for a certain amount of time. An appointment can be recurrent, in other words, it repeats regularly over time (e.g., lunch, training, etc.). User can travel with someone else and can pick up or leave off these people during the day.

User can have his own travel means and a pass for public transportation. 
The travel means considered in this scenario can be grouped in three categories: public, shared or private.
\begin{itemize}
\item Public travel means: these include trains, buses, underground, taxis, trams. They have to be taken in their \textbf{appositi} stops. User must have a valid ticket in order to get on a public travel means (except for taxis, that pick up the user wherever he wants upon a call and do not require any ticket). 
\item Shared travel means: these include car and bike. They are located in specific places and require a reservation in order to be used by the user.
\item Private travel means: vehicles owned by the user. They can be cars, bikes, motorbikes.
\end{itemize}

Weather conditions can change during the day affecting usable travel means.
At the beginning of the day, or on demand, user can request a schedule of his daily appointments, following some criteria evaluated according to their assigned priority and satisfying some constraints imposed by the user.
When a new appointment is received, user creates a new item in the application and saves it in the appointment list. User can request a reschedule to the application due to unexpected changes of his plan (e.g. a cancelled appointment).

\subsection{World Phenomena}
\begin{itemize}
\item User receives a new appointment
\item User has to travel alone or with someone else
\item User owns private travel means and/or passes for public transportation
\item User wakes up
\item User pass expires
\end{itemize}

\subsection{Shared Phenomena}
\begin{itemize}
\item Shared travel mean moves
\item Shared travel mean its not available anymore
\item Wheather condition changes
\item Public travel means reach a stop-place
\item Public travel means are late 
\item User requests a schedule to the machine
\item User inserts a new appointment into the application
\item User requests to book rides
\item User moves
\end{itemize}

\section{Definitions, Acronyms, Abbreviations}

Appointment
Constraint
Criteria
GPS
Schedule/Scheduler
System/Applications

GUI: graphic user interface

%\reqref{req:R1}
%\goalref{goal:G1}

\section{Revision history}

\section{Reference documents}

\section{Document structure}