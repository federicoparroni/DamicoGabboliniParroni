\chapter{Specific requirements}

\section{External Interface Requirements}
The application shows its best potential when run in a mobile device, for instance a smartphone or a tablet. This permits to extend the features and the automatic tasks of the application, thanks to the built-in device functionalities. However, a computer client version of the application can be installed, too.

\subsection{User interfaces}
The user can interact with the application through several graphical interfaces:
\begin{enumerate}
\item Registration/login interface: allows the user to insert credentials in order to registering or logging into the system;
\item User profile interface: user can specify his characteristics, such as his passes, car and/or bike ownership;
\item Appointment CRUD interface: allows creating, showing and editing appointment parameters and related constraints;
\item Non-scheduled appointments interface: provides a list of all inserted appointments, but not \textbf{already} scheduled (includes the possibility to delete an item of the list);
\item Schedule interface: user can set optimization criteria and request a schedule creation for a given date
\item Schedule result interface: shows the computation of the requested schedule and permits to keep track of the completeness percentage, indicating the directions to be followed by the user in a map, in order to arrive to the next appointments
\end{enumerate}


\subsection{Hardware interfaces}
The system relies on several hardware architectures:
\begin{itemize}
\item Mobile device:
\item Servers: APIs
\item Travel means: gps on taxi and shared mean
\end{itemize}

\section{Functional requirements}
\subsection{Scenarios}
Here are some scenarios that describe the usage of the system.
\subsubsection{Scenario 1} \label{scenario:1}
Edoardo has headache in the morning so he decides to call his doctor and they decide to meet the same day. The doctor tells him to come to his studio between 3pm and 4pm. Edoardo decides to go at 3pm 
