\chapter{Specific requirements}

\section{External Interface Requirements}
The application shows its best potential when run in a mobile device, for instance a smartphone or a tablet. This permits to extend the features and the automatic tasks of the application, thanks to the built-in device functionalities. However, a computer client version of the application can be installed, too.

\subsection{User interfaces}
The user can interact with the application through several graphical interfaces:
\begin{enumerate}
\item Registration/login interface: allows the user to insert credentials in order to registering or logging into the system;
\item User profile interface: user can specify his characteristics, such as his passes, car and/or bike ownership;
\item Appointment CRUD interface: allows creating, showing and editing appointment parameters and related constraints;
\item Non-scheduled appointments interface: provides a list of all inserted appointments, but not \textbf{already} scheduled (includes the possibility to delete an item of the list);
\item Schedule interface: user can set optimization criteria and request a schedule creation for a given date
\item Schedule result interface: shows the computation of the requested schedule and permits to keep track of the completeness percentage, indicating the directions to be followed by the user in a map, in order to arrive to the next appointments
\end{enumerate}


\subsection{Hardware interfaces}
The system relies on several hardware architectures:
\begin{itemize}
\item Mobile device:
\item Servers: APIs
\item Travel means: gps on taxi and shared mean
\end{itemize}

\section{Functional requirements}
\subsection{Scenarios}
Here are some scenarios that describe the usage of the system.
\subsubsection{Scenario 1} \label{scenario:1}
Giovanni will start the fourth year of his Master's degree. Surfing the internet, he finds out that his lesson schedule for the first semester it has been published. Giovanni decides to fill in the application with his new appointments concerning the attendance of lessons. In fact he knows where to go, at which time and day and for which amount of time. Since he knows that these events will going to happen for 3 months, he sets that they should be recurrent

\subsection{Use cases}

\subsubsection{Appointment creation}

\begin{table}[]
\centering
\caption{}
\begin{tabular}{ll}
\hline
\textbf{Name}   & \textbf{Appointment creation}   \\ \hline
\textbf{Actors} &  User \\ \hline
\textbf{Goals} &  \goalref{goal:G1} \\ \hline
\textbf{Input Condition} &   \\ \hline
\textbf{Event Flow} &  \begin{enumerate}
\item The user wants to add a new appointment to his schedule
\item The user logs-in in the application (\textbf{ref to the log-in use case}.
\item The 
\end{enumerate}	
\end{enumerate}  \\ \hline
\end{tabular}
\label{my-label}
\end{table}

\subsubsection{Notes on use cases}
eventuali informzioni che specificano meglio come leggere i deagrammi che si producono
