\chapter{Specific requirements}

\section{External Interface Requirements}
The application shows its best potential when running in a mobile device, for instance a smartphone or a tablet. This permits to extend the features and the automatic tasks of the application, thanks to the built-in device functionalities. However, a computer client version of the application can be installed, too.

\subsection{User interfaces}
The user can interact with the application through several graphical interfaces:
\begin{enumerate}
\item \textbf{Registration/login interface}: allows the user to insert credentials in order to registering or logging into the system;
\begin{figure}[H]
\begin{center}
\includegraphics[width=250pt, keepaspectratio]{"images/interfaces/login".jpg}
\caption{Registration/login interface}
\end{center}
\end{figure}
\item \textbf{User account interface}: user can specify his profile characteristics, such as his passes, car and/or bike ownership;
\begin{figure}[H]
\begin{center}
\includegraphics[width=250pt, keepaspectratio]{"images/interfaces/userprofile".jpg}
\caption{User account interface}
\end{center}
\end{figure}
\item \textbf{Home interface}: shows currently running schedule and displays some navigation links to other interfaces;
\begin{figure}[H]
\begin{center}
\includegraphics[width=250pt, keepaspectratio]{"images/interfaces/"home".jpg}
\caption{Home interface}
\end{center}
\end{figure}
\item \textbf{Appointment CRUD interface}: allows creating, showing and editing appointment parameters and related constraints;
\begin{figure}[H]
\begin{center}
\includegraphics[width=250pt, keepaspectratio]{"images/interfaces/appointment".jpg}
\caption{Appointment CRUD interface}
\end{center}
\end{figure}
\item \textbf{Appointments interface}: provides a list of all inserted appointments, with the possibility to filter between non-scheduled/scheduled ones (includes the possibility to delete an item of the list);
\begin{figure}[H]
\begin{center}
\includegraphics[width=250pt, keepaspectratio]{"images/interfaces/"appointments".jpg}
\caption{Appointments list interface}
\end{center}
\end{figure}
\item \textbf{Schedule interface}: user can set parameters, contraints, optimization criteria and request a schedule creation for a given date;
\begin{figure}[H]
\begin{center}
\includegraphics[width=250pt, keepaspectratio]{"images/interfaces/"schedule".jpg}
\caption{Schedule interface}
\end{center}
\end{figure}
\item \textbf{Schedules result interface}: shows the computation of the requested schedules for a given date and asks the user to select one, then waits for confirmation for that;
\item \textbf{Schedule progress interface}: permits to keep track of the completeness percentage, indicating the directions to be followed by the user in a map, in order to arrive to the next appointments;
\item \textbf{Tickets/rides reservation interface}: allows user to buy tickets for public travel means and/or reserve a ride for the shared travel means;
%\item \textbf{Appointments history interface}: shows a list of archived appointments;
\end{enumerate}

\subsection{Hardware interfaces}
Hardware interfaces are physical linking across which two or more separate components of a system exchange information. A hardware interface is described by the mechanical and electrical signals at the interface and the protocol for sequencing them. There are no interesting hardware interfaces in our scope.
%Our system relies on the following hardware interfaces:
%\begin{itemize}
%\item Mobile device: 
%\item Server: the subsystem is based on the client-server paradigm. 
%\item Travel means: gps on taxi and shared mean
%\end{itemize}

\subsection{Software interfaces}
Software interfaces are logical linking across which two or more separate applications running on a system exchange information. The most relevant software interface in our system is API. APIs are sets of subroutine definitions, protocols and clearly defined methods of communication, allowing data exchanging and service requests. There are several kinds of these:
\begin{itemize}
\item \textbf{Operating System APIs}: specify interface between applications and OS, permitting to access low level routines calls (for instance, to communicate with memory or with an internal device)\textbf{(!)}
\item \textbf{Remote APIs}: DBMS expose a set of standards that the API user can adopt in order to manage the database data. SQL is the standard language for storing, manipulating and retrieving data in this context;
\item \textbf{Web API}: information can be exchanged through the internet by encapsulating it in HTTP request/response. Weather forecast, travel services, mapping systems offer this typology of API.
\end{itemize}

\subsection{Communications interfaces}
Communication interfaces allows two different architectures of the system to exchange information through communication channel. These non-homogeneous components of the system can communicate thanks to the following software interfaces and protocols:
\begin{itemize}
\item \textbf{Cellular connectivity}: mobile devices can connect to the internet thanks to LTE standard;
\item \textbf{GPS}: cellular can retrieve his coordinates position through NMEA protocol;
\item \textbf{QRCode}: associates a matrix of bits to an URL. QRCodes are present in most of the shared means, semplifying the booking of that.
%identify the nearby transportation
\end{itemize}


\section{Functional requirements}
\subsection{Scenarios}

Here are some scenarios that describe the usage of the system.

\subsubsection{Scenario 1} \label{scenario:1}
Giovanni will start the fourth year of his Master's degree. Surfing the internet, he finds out that his lesson schedule for the first semester it has been published. Giovanni decides to fill in the application with his new appointments related to lessons attendance. In fact he knows where to go, at which time and day and for which amount of time. Since he knows that these events will going to happen for 3 months, he sets them as recurrent.

\subsubsection{Scenario 2} \label{scenario:2}
\textbf{da rileggere scritta veloce}
Giovanni want to start training but he doesn't know what are the best hours in which he can run in accord to his appointments, he know only that he can run between 5 and 7 pm, for 45 minutes. he can insert this last appointment in the application whitout specify the exactly starting hour and the system will give him the best hours in which he can run

\subsubsection{Scenario 3} \label{scenario:3}
\textbf{da rileggere scritta veloce}
Giovanni has scheduled his appointments but at lunch time his son called him because he needed a ride for go back to home. Giovanni decided to help his son and so he brought him home. now the current running schedule is not more valid so he request to the system a reschedule of his appointment according to his position and the hour of day in which he is.  

\subsection{Use cases}

\subsubsection{User log-in}
\begin{tabular}{|p{14cm}|} \hline

\textbf{Name:} User log-in \\ \hline
\textbf{Actors:} User \\ \hline
\textbf{Goals}: (\textbf{Goal del login} \\ \hline
\textbf{Input Condition:} The user is registered to the system \\ \hline
\textbf{Event Flow:} 
\begin{enumerate}
\item The user needs to log-in the application, so he runs it;
\item The system provides to the user a form to fill;
\item The user fills up the form with the his e-mail and his password (as said in \ref{subsect:usermodel})
\item The user submits the form to the system;
\item The system checks the user identity and provides to the user the main application page (\textbf{reference to the main application page})
\end{enumerate} \\ \hline

\textbf{Output Condition:} The user is logged-in to the system. \\ \hline

\textbf{Exceptions:} The user submits the form after having filled it with a wrong email or password. \\ \hline

\textbf{Mapping on requirements:}
\begin{itemize}
\item Events from 3 through 5 granted by (\textbf{requirement che può recuperare informazioni dell'utente riguardo i dati della registrazione});
\item Event 6 grandet by (\textbf{requirement che il sistema può controllare se l'identità di un utente è giusta};
\end{itemize} \\ \hline

\end{tabular}

\includegraphics[width=400pt, keepaspectratio]{"images/SequenceDiagramLogIn".png}

\subsubsection{Appointment creation} \label{usecase:appcreation}

\begin{longtable}{|p{14cm}|} \hline
\textbf{Name:} Appointment creation \\ \hline
\textbf{Actors:} User \\ \hline
\textbf{Goals:} \goalref{goal:G1} \\ \hline
\textbf{Input Condition:} 
\begin{itemize}
\item The user is registered to the system 
\item The user is logged in to the systems 
\end{itemize}
\\ \hline
\textbf{Event Flow:}
\begin{enumerate}
\item The user wants to add a new appointment to his schedule;
\item The user requests the appointments page
\item The system provides the appointments page
\item The user requests the creation of a new appointment to the application;
\item The system provides to the user a form to fill;
\item The user fills up the form with the parameters (specified in \ref{subsect:appointmentmodel}) and constraints (specified in \ref{subsubsect:constronappoint} about the new appointment;
\item The user submit the form to the system;
\item The system allocates the new appointment as Unscheduled (referring to statechart in figure; \label{fig:stchartApp})
\item The system sends a confirmation to the user.
\end{enumerate}	\\ \hline

\textbf{Output Condition:} The user has created a new appointment; \\ \hline

\textbf{Exceptions:}
\begin{enumerate}
\item Some fields of the form referring to parameters are left blank;
\item The \textit{location} field doesn't belong to the domain area of the application (\textbf{riferimento alla domain assumption della regione})
\end{enumerate} \\ \hline
\textbf{Mapping on Requirements:}
\begin{itemize}
\item Events 4 through 7 are granted by (\textbf{requirement che il sistema può recuperare informazioni riguardanti un appointment})
\item Event 7 is granted by (\textbf{requirement che il sistema è in grado di memorizzare un appointment}
\end{itemize}  \\ \hline


\end{longtable}

\begin{figure}[H]
\begin{center}
\includegraphics[width=400pt, keepaspectratio]{"images/sequenceDiagramAppointmentCreation".png}
\caption{Appointment creation sequence diagram}
\label{img:seqDiagrAppCreation}
\end{center}
\end{figure}

In the sequence diagram there's the assumption that the log-in proceeds successfully. The log-in procedure referenced is the one explained in (\textbf{use case del log-in}.)

\subsubsection{ScheduleSelection}
\begin{longtable}{|p{14cm}|} \hline
\textbf{Name:} Multiple Schedules creation \\ \hline
\textbf{Actors:} User \\ \hline
\textbf{Goals:} \textbf{aggiungere rif al goal} \\ \hline
\textbf{Input Condition:} 
\begin{itemize}
\item The user is registered to the system 
\item The user is logged in to the systems 
\end{itemize}
\\ \hline
\textbf{Event Flow:}
\begin{enumerate}
\item The user wants to compare multiple schedules;
\item The user requests the schedules page;
\item The system provides the schedules page;
\item The user selects a schedule to be run;
\item The system display the mainpage with the schedule results (fare riferimento alla definizione)
\end{enumerate}	\\ \hline
\textbf{Output Condition:} The user selects a schedule to be run \\ \hline
\textbf{Exceptions:}
\\ \hline
\textbf{Mapping on Requirements:}
\begin{itemize}
\item Events are granted by the requirment R10 \textbf{(mettere riferimenti)}
\end{itemize}  \\ \hline

\end{longtable}
\label{usecase:ScheduleSelection}

\begin{figure}[H]
\begin{center}
\includegraphics[width=400pt, keepaspectratio]{"images/ScheduleSelectionSequenceDiagram".png}
\caption{Appointment creation sequence diagram}
\label{img:ScheduleSelection}
\end{center}
\end{figure}

\subsubsection{Appointment editing}\label{usecase:appediting}
\begin{longtable}{|p{14cm}|} \hline
\textbf{Name:} Appointment editing \\ \hline
\textbf{Actors:} User \\ \hline
\textbf{Goals:} (\textbf{goal che l'utente può modicare un appointment})\\ \hline
\textbf{Input Condition: The user is logged-in to the system} \\ \hline
\textbf{Event Flow:}
\begin{enumerate}
\item The user wants to modify an appointment of his schedule;
\item The user selects the appointment to modify;
\item The system provides to the user the appointment form with all the parameters and constraints (with reference to %\ref{subsect:appointmentmodel 
and 
%\ref{subsect:appointmentmodel}} 
that were specified yet by the user; 
\item The user edits the fields of the form;
\item The user submit the form to the system;
\item The system set the appointment as Unscheduled with the new parameters (referring to statechart in figure ..); \label{fig:stchartApp})
\item The system sends a confirmation to the user.
\end{enumerate}	\\ \hline
\textbf{Output Condition:}  The user has modified an appointment; \\ \hline
\textbf{Exceptions:}
\begin{enumerate}
\item Some fields of the form referring to parameters are left blank;
\item The \textit{location} field doesn't belong to the domain area of the application (\textbf{riferimento alla domain assumption della regione})
\end{enumerate} \\ \hline
\textbf{Mapping on Requirements:}
\begin{itemize}
\item Events 3 through 5 are granted by \textbf{(the requirement that says that the system should let the user change the parameters and the constraints of an inserted appointment)}
\item Event 6 is granted by (\textbf{requirement che il sistema è in grado di memorizzare un appointment modificato}
\end{itemize}  \\ \hline

\end{longtable}

\begin{figure}[H]
\begin{center}
\includegraphics[width=400pt, keepaspectratio]{"images/seqDiagramAppointmentEditing".png}
\caption{Appointment editing sequence diagram}
\label{img:seqDiagrAppEditing00}
\end{center}
\end{figure}


\subsubsection{Schedule appointments}\label{usecase:scheduleappointments}
\begin{longtable}{|p{14cm}|} \hline
\textbf{Name:} Schedule appointments \\ \hline
\textbf{Actors:} User, External API \\ \hline
\textbf{Goals:} (\textbf{goal che permette di fare uno scheduling degli appuntamenti dell'utente})\\ \hline
\textbf{Input Condition: The user is logged-in to the system (\textbf{in futuro mettere che l'actor è un logged user e togliere questo, se ci piace di più}} \\ \hline

\textbf{Event Flow:}
\begin{enumerate}
\item The user wants to schedule his appointments;
\item The user requests a schedule by mean of selecting the appointments to schedule of the current day;
\item The system provides to the user the schedule form with all the parameters, optimization criteria (with reference to (\textbf{parametri dello schedule} and \textbf{opt criteria dello schedule}) and the constraints (\textbf{riferimento ai constraints sullo schedule}.
\item The user fills up the fields of the form;
\item The user submits the form to the system;
\item The system retrieves information from external APIs about; travel options and related travel option data, weather forecast and strike days;
\item The system retrieves about the Travel Option Data of the newly created Schedule
\item The system stores the Schedule, together with his travel option data

(\textbf{link alla definizione}) and stores it as Saved (\textbf{fare riferimento allo state chart}) with the appointment selected by the user.
\end{enumerate}	\\ \hline

\textbf{Output Condition:} The user has created a valid schedule of his appointments; \\ \hline

\textbf{Exceptions:}
\begin{enumerate}
\item Some fields of the form referring to schedule variables and optimization criteria are left blank. The parameters of schedule constraints could also be left blank since they will assume default values;
\item It's not possible to list the appointments as a Valid Schedule, so the schedule is Discarded \textbf{(riferimento allo statechart)}
\end{enumerate} \\ \hline
\textbf{Mapping on Requirements:}
\begin{itemize}
\item Events 2 through 5 are granted by \textbf{(the requirement that says that the user can set the opt criteria, constraints, variables and  of the schedules)};
\item Event 6 and 7 is granted by (\textbf{the requirement that says that we can retrieve info from APIs});
\item Event 8 is granted by (\textbf{the requirement that says that is possible to store a schedule}, (\textbf{requirement che il sistema è in grado di ottimizzare in base a quanto gli dico}) 
\end{itemize}  \\ \hline

\end{longtable}

\begin{figure}[H]
\begin{center}
\includegraphics[width=450pt, keepaspectratio]{"images/SequenceDiagramSchedule".png}
\caption{Schedule appointents sequence diagram}
\label{img:seqDiagrAppEditing00}
\end{center}
\end{figure}


\subsubsection{Recover credentials}\label{usecase:recovercredentials}
\begin{longtable}{|p{14cm}|} \hline
\textbf{Name:} Schedule appointments \\ \hline
\textbf{Actors:} User, External Email Service \\ \hline
\textbf{Goals:} (\textbf{goal che permette di recuperare le credenziali dell'utente})\\ \hline
\textbf{(\textbf{in futuro mettere che l'actor è un registered user e togliere questo, se ci piace di più}} \\ \hline

\textbf{Event Flow:}
\begin{enumerate}
\item The user wants to recover his passoword;
\item The user requests to recover his passoword;
\item The system provides to the user the schedule form with his e-mail;
\item The user fills up the field of the form;
\item The user submits the form to the system;
\item The system sends the e-mail to the user with his password.
\end{enumerate}
\\ \hline

\textbf{Output Condition:} The user has recovered his password; \\ \hline

\textbf{Exceptions:}
\begin{enumerate}
\item The form it's left blank;
\item An invalid address is given;
\end{enumerate} \\ \hline

\textbf{Mapping on Requirements:}
\begin{itemize}
\item Actions 2 through 4 granted by requirement (\textbf{the system should be able to recover email address from the user}
\item Action 5 granted by the other requirement
\end{itemize}  \\ \hline

\end{longtable}

\begin{figure}[H]
\begin{center}
\includegraphics[width=450pt, keepaspectratio]{"images/PasswordRecoverySequenceDiagram".png}
\caption{Password recovery sequence diagram}
\label{img:seqDiagrPasswordRecovery}
\end{center}
\end{figure}



\subsection{Definition of use case diagrams}
In these diagrams we assume that the user has ran the application. In particular, if the actor is a \textit{Registered User} or an \textit{Enternal User} then we assume that the user is facing the login page (\textbf{riferimento alla login page}. On the other hand, if the actor is a \textit{Logged User} then is assumed that it is on the home page (\textbf{riferimento alla home page}.