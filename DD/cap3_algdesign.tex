\chapter{Algorithmic Design}
In this chapter are given the guidelines on how to implement the most important functionalities that the components of the system will offer. The pseudocode of the relevant method is shown. 

\section{Login}
The Login involves the application the application server and the database. The last one is located outside the application server. 
To login in to the application the following steps must be done:
\begin{enumerate}
\item the user presses the Login button in the Login view;
\item the Identity manager (Login method) sends a request to the Application Server for authentication
\item A query on the database is performed to check the validity of the user credential and a response to the client is returned;
\end{enumerate}

\begin{lstlisting}
//client side
Login(username, password)
	response = SendRequest("api/login", username, password)
	if response.isValid 
		token = response.GetToken
		Show(HomeView)
	else
		Show(ErrorMessage , "Invalid Credentials")

//server side
LoginRequest(username, password)
	result = database.query(username, password)
	if result == 1
		SendAuthenticationResponse (token)
	else
		sendAuthenticationError(error code)
\end{lstlisting}

\section{CreateSchedule}
\textbf{correggere formula}
The process of computing a schedule is composed by the following steps:

\begin{enumerate}
\item Create the predecessor matrix (P) in which every cell (i,j) contains:
\begin{equation}
    P_{ij}=
    \begin{cases}
      1, & if Appointment_i must precede Appointment_j \\
      0, & if Appointment_i must follows Appointment_j \\
      -1, & if Appointment_i can be scheduled both before or after appointment_j
    \end{cases}
  \end{equation}
\item Compute all the possible ordered arrangements of appointments with respect to the predecessor matrix
\item For each appointment in each arrangement, set the starting time according to travel time. This is estimated considering the euclidean distance and heuristics on the kind of travel mean between every pair of consecutive appointments in the arrangement, considering the bestTravelMean(\textbf{fare def});
\item Check the feasibility of the arrangements and discard the ones which have overlapping appointments;
\item Choose the most convenient one, according to the optimization criteria.
\item Call mapping service Api to fix the effective routes between the appointments
\end{enumerate}

\begin{lstlisting}
ComputeSchedule(wakeUpTime, startingLocation, appts, constraints, optCriteria)
	p=CalculatePredecessorMatrix(appts)
	a=CalculateArrangements(appts, p, 0, 1)
	SetStartingTime(a, startingTime, startingLocation, constraint, optCriteria)
	s=ChooseBestSchedule(a)
	MappingServiceRequest(s)
	return s

CalculatePredecessorMatrix(appts)
	p=new Matrix[appt.size,appt.size]
	for(i=0 .. appts.size)
		for(j=i+1 .. appts.size)
			a1=appts[i]
			a2=appts[j]
			if a1.deterministic && a2.deterministic
				if a1.startingTime < a2.startingTime
					pred[i,j]=1
				else
					pred[i,j]=0
			elseif a1.deterministic && !a2.deterministic
				if a1.startingTime < a2.timeSlot.start
					pred[i,j]=1
				elseif a1.endingTime > a2.timeSlot.end
					pred[i,j]=0
				else
					pred[i,j]=-1
			elseif !a1.deterministic && a2.deterministic
					if a1.timeslot.end < a2.endingTime
						pred[i,j]=1
					elseif a1.timeSlot.start > a2.startingTime
						pred[i,j]=0
					else
						pred[i,j]=-1
			elseif !a1.deterministic && !a2.deterministic
					if a1.timeSlot.end < a2.timeSlot.start
						pred[i,j]=1
					elseif a1.timeSlot.start > a2.timeSlot.end
						pred[i,j]=0
					else
						pred[i,j]=-1
	return p
		
		
CalculateArrangements(appts, p, curri, currj)
	arrangement=new List	
	for(i=curri .. appts.size-1)
		for(j=currj-1 .. appts.size)
			if p[i,j]==-1
				p0=p
				p0[i,j]=0
				CalculateArrangements(appts, p0, i, j)
				
				p1=p
				p1[i,j]=1
				CalculateArrangements(appts, p1, i, j)
				
				return
				
	a=ConvertPredMatrixToList(appts,p)
	arrangement.addLast(a)
	return arrangement

		 
ConvertPredMatrixToList(appts, p)
//converts a "-1 free" predecessor matrix to an ordered list of appointments


SetStartingTime(a, startingTime, startingLocation, constraint, optCriteria)
	for(arr in a)
		dummyStartingAppt = new appointment(startingTime, startingLocation, duration=0)
		arr.addFirst(dummyStartingAppt)
		for(i=1 .. arr.size)
			appt1=arr[i-1]
			appt2=arr[i]
			travelMean=getBestTravelMean(appt1, appt2, constraint, optCriteria)
			travelTime=travelMean.estimateTime(appt1, appt2)
			if appt2.deterministic 
				appt2.startingTravelTime = appt2.startingTime-travelTime
				appt2.travelMean=travelMean
				if appt1.endingTime > appt2.startingTravelTime
					error("schedule not feasible")
			else
				appt2.startingTravelTime = max(appt1.endingTime,appt2.timeSlot.start-travelTime)
				appt2.travelMean=travelMean
				if appt2.startingTravelTime > appt2.timeSlot.end
					error("schedule not feasible")
					
					
getBestTravelMean(appt1, appt2, constraint, optCriteria)
	l=getNotConstrainedTravelMeans(constraint)
	for(t in l)
		switch optCriteria
			case "MoneySpent"
				t.cost=estimateMoney(t, appt1, appt2)
			case "Time"
				t.cost=estimateTime(t, appt1, appt2)
			case "CarbonFootprint"
				t.cost=estimateCarbon(t, appt1, appt2)
	sortByCriteria(l, optCriteria)
	return l
	
ChooseBestSchedule(a)
	best=a[0]
	for(i = 0 .. a.size)
		sum=0
		for(appt in a[i])
			sum+=appt.travelMean.cost
		a[i].totalCost=sum
		if sum<a[0].totalCost
			best=a[i]
	return best
	
MappingServiceRequest(s)
	for (i=0 .. s.size-1)
		response=MappingServiceRequest(s[i], s[i+1])
		s[i].path=response.path
		s[i].startingTime=response.startingTime

\end{lstlisting}

\section{Registration}
The Registration involves the application the application server and the database. The last one is located outside the application server. 
To register in to the application the following steps must be done:
\begin{enumerate}
\item The user presses the Registration button in the Login view;
\item The Identity manager (Registration method) sends a request to the Application Server for authentication
\item A query on the database is performed to check the presence of the user credential and a confirmation email is sent
\item The user confirms the email by clicking on the designated link
\item The user's state on the database becomes confirmed
\end{enumerate}

\begin{lstlisting}
//client side
Register(username, password)
	response = SendRequest("api/registration", username, password)
	if !response.valid
		Show(ErrorMessage , "User Already Registered")
		

	
//server side
RegistrationRequest(username, password)
	result = database.query(username, password)
	if result == 0
		database.insertTuple(username, password)
		sendEmail(username)
	else
		sendAuthenticationError(error code)
		
EmailConfirmationRequest(username)
	result = database.modifyTuple(username, confirmed=true)
\end{lstlisting}


\section{Synchronize}
The synchronization in our system is the process that aims to keep the data consistent and updated between different devices. For example, when a user inserts an appointment on one of his devices, then the changes must be propagated to all his other devices, once the login is performed. 
The Synchronization involves the application, the application server and the database. The last one is located outside the application server. To sinchronize data across multiple devices two actions must be carried out:
\begin{itemize}
\item Upload of local data on the database when a single change on the  client's local data occours(Synchronize Upwards);
\item Download of data from the database, if an update is necessary, when login is performed(Synchronize Downwards).
\end{itemize}

\begin{lstlisting}
//client background processes

SynchronizeUpwards(changedData)
	update = false
	while !update
		response = SendRequest("api/sync/up", token, changedData)
		if response	 
			update = true
			
SynchronizeDownwards()
	newData=SendRequest("api/sync/down", token)
	localData=newData
		
		
//server side

SynchUpwardsRequest(token, changedData)
	userID = database.getUser(token)
	if changedData.action == insert
		//changedData is a newly inserted element
		database.insert(changedData, userID)
	else if changedData.action == edit
		database.update(changedData, userID)
	else
		database.delete(changedData, userID)
	sendResponse("syncresult", true)
	
SynchUpwardsRequest(token, changedData)
	userID = database.getUser(token)
	data=database.getUserData(userID)
	sendResponse(data)

\end{lstlisting}




		