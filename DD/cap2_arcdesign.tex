\chapter{Architectural Design}

\section{Overview}
Travlendar+ has a multi-tier architecture.

\begin{figure}[H]
\begin{center}
\includegraphics[width=\textwidth, keepaspectratio]{"image/components".png}
\caption{High level components}
\end{center}
\end{figure}

User can interact  directly with both version of the application: desktop and mobile. The schedule computation is performed locally on the device in which the application runs by querying the external APIs:

\begin{itemize}
\item Mapping Service API
\item Travel Means APIs
\item Wheather Forecasts APIs
\end{itemize}

For the last two ones, the communication happens through different interfaces (wrappers), in order to improve the capability of the system to be expanded. In this way we provide a common pattern for the data acquired from the APIs, so that new ones can be easily included without the need of drastic changes on the central core.

\section{Component View}

\section{Deployment View} 

\section{Runtime View}

\section{Component Interfaces}

\section{Selected architectural styles and patterns}

\section{Other Design Decisions}