%%CAPITOLO 1: Introduzione=======================================
\chapter{Introduction}

\section{Purpose}

\section{Scope}

Travlendar+ is an application that aims to ease the management of the daily appointments of a registered user by providing a desired schedule for those. 
The application runs within the context of a restricted geographic area and thus should consider only the travel means available in this scope.
The application is able to interact with different data sources, in particular Travel Means APIs \textbf{FARE DEFINIZIONE} and User device GPS, and to arrange the appointments by exploiting these information.
Once the schedule is computed, the application gives to the user the possibility to buy tickets for the involved travel means, then he/she can run it and follow directions provided by the application. 

\section{Definitions, Acronyms, Synonims}

\subsection{Synonims}
%Are synonims:
%\begin{itemize}
%\item Appointment and meeting;
%\item System and Application.
%\end{itemize}

\subsection{Definitions}
\theoremstyle{definition}
\newtheorem{definition}{Definition}[section]
 
\begin{definition} \label{def:userdata}
the user data include:
\begin{itemize}
\item schedules;
\item appointments;
\item user parameters.
\end{itemize}

\end{definition}
  
%\begin{definition} \label{def:preference}
%A preference is a constraint on appointment or a schedule;
%\end{definition}
%
\begin{definition} \label{def:device}
A device is a PC, a Tablet or a Smartphone in which run the last version of his O.S.;
\end{definition}

%\begin{definition} \label{def:travelOption}
%A Travel Option is a combination of travel path and travel means that allow to reach one spot from another;
%\end{definition}
%
%\begin{definition} \label{def:travelOptionData}
%The Travel Option Data are additional information about a travel option:
%\begin{itemize}
%\item Cost;
%\item Traveling time;
%\item Carbon emission;
%\item Distance (KM);
%\item Graphical representation of the path.
%
%\end{itemize}
%\end{definition}
%
\begin{definition} \label{def:schedule}
A Schedule is a set of time-ordered and not overlapping appointments where their starting times are fixed and they're linked to each other by a path covered with a specific transportation mean;
\end{definition}

%\begin{definition} \label{def:validSchedule}
%A Valid Schedule is a Schedule which:
%\begin{itemize}
%\item Is optimized according to the criteria chosen by the user;
%\item Ensures that the user will be on time for all his appointments;
%\item Respects the constraints imposed by the user;
%\end{itemize}
%\end{definition}

%\begin{definition} \label{def:travelServiceAccount}
%A travel service account is an external account of the user which permits the booking and the payment of a specific travel mean;
%\end{definition}
%
%\begin{definition} \label{def:relativePath}
%A Relative Path is a  portion of a path travelled by the same travel means;
%\end{definition}
% 
%
%\begin{definition} \label{def:schedulingResult}
%A Scheduling Result is the set:
%\begin{itemize}
%\item Graphical representation of the path that will be travelled by the user
%\item Money spent for each relative path
%\item Total money spent 
%\item Length of the path expressed in KM
%\item Length of relative path 
%\item Carbon footprint emission
%\item Estimated travel duration of each relative path
%\item Total estimated travel time
%\end{itemize}
%\end{definition}

%\begin{definition} \label{def:currentAppointment}
%The Current Appointment is an appointment which has \textit{startingTime >= currentTime } and \textit{date=currentDate}, where \textit{currentTime} and  \textit{currentDate} are the actual time reference of the system.
%\end{definition}

%convenzioni:
%variables are italic
%states are bold

\subsection{Acronyms}
%Acronyms used in the text:
\begin{itemize}
%\item GPS: Global Positioning System;
%\item GUI: Graphical User Interface;
%\item ETA: Estimated Time of Arrival;
%\item S.P.W.: Should Provide a Way;
%\item S.B.A.: Should Be Able;
\item API: Application Programming Interface;
%\item CRUD: Create/Read/Update/Delete;
%\item URL: Uniform Resource Locator;
%\item O.S.: Operating System.
\end{itemize}

%\reqref{req:R1}
%\goalref{goal:G1}

\section{Revision history}

\section{Document structure}


 
