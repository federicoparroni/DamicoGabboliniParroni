\documentclass [11pt,a4paper,oneside,openany]{book} %Classe del documento, formato carta, singola facciata, apertura capitoli pag destra/sinista (ininfluente se impostato oneside). Formato libro
\usepackage[italian]{babel} %Lingua Documento (da impostare per la sillabazione)
\usepackage{graphicx} %Pacchetto necessario per la gestione delle immagini
\usepackage{xcolor,colortbl} %Colori
\usepackage{enumitem} % pacchetto per gestire le enumerazioni e resume
\usepackage{fancyhdr} %Pacchetto per la gestione accurata della pagina
\usepackage{setspace} %Pacchetto necessario per i comandi successivi onehalfspacing, singlespacing....
\usepackage{titlesec} % Pacchetto per le sezioni e sottosezioni
\usepackage{longtable} %Pacchetto per la gestione delle tabelle grandi
\usepackage[colorlinks=true]{hyperref} %Segnalibri nel pdf finale, sotto relativi parametri
\usepackage[T1]{fontenc}
\usepackage[utf8]{inputenc} %Pacchetti per le lettere accentate utf8
\usepackage{listings} %Pacchetto per formattare codice sorgente
\usepackage{mathtools} %Pacchetto matematica
\usepackage{float} %Pacchetto per il posizionamento delle figure
\usepackage{amsthm}

\hypersetup{
	bookmarksnumbered=true,
	linkcolor=black,
	citecolor=black,
	urlcolor=black,
}

\usepackage{geometry} % Dimensione pagina

\geometry{a4paper} % Formato carta
\addtolength{\textheight}{75pt} %Margini
\oddsidemargin 30pt

\setcounter{secnumdepth}{2}
\setcounter{tocdepth}{3}


\newcommand{\reqref}[1]{\textbf{R\ref{#1}}}
\newcommand{\goalref}[1]{\textbf{G\ref{#1}}}

\definecolor{darkgreen}{rgb}{0,0.6,0}
\lstset{frame=tb,
  language=Java,
  aboveskip=3mm,
  belowskip=3mm,
  showstringspaces=false,
  columns=flexible,
  basicstyle={\small\ttfamily},
  numbers=none,
  numberstyle=\tiny\color{gray},
  keywordstyle=\color{blue},
  commentstyle=\color{darkgreen},
  %stringstyle=\color{mauve},
  breaklines=true,
  breakatwhitespace=true,
  tabsize=3
}

\begin{document} %Inizio Documento

%%FRONTESPIZIO%%=========================================
\begin{titlepage}

 \begin{center} 
     \includegraphics[height=6cm]{logo.eps}\\
     \vspace{4em}
     {\Large \textsc{Software Engineering II}}\\
     \vspace{6em}
     {\LARGE \textbf{Travlendar+}}\\
     \vspace{3em}
     {\Large \textsc{Design}}\\
     \vspace{1em}
     {\Large \textsc{Document}}\\
 \end{center}
 
    \vskip 2cm
 
 	Authors:
 	\vspace{0.5em}
 	\begin{center}
      {\Large \textit{Edoardo D'Amico}}\\
      {\Large \textit{Gabbolini Giovanni}}\\
      {\Large \textit{Parroni Federico}}\\
    \end{center}

\vskip 2.0cm
\begin{center}
{\normalsize 1$^{st}$ November 2017}
\end{center}

\end{titlepage}

\newpage

%%INTESTAZIONI PAGINA%%==================================
\pagestyle{fancy}
\renewcommand{\chaptername}{Chapter}
\renewcommand{\chaptermark} [1]{\chaptername\ \thechapter.\ #1}{} 
\renewcommand{\chaptermark}[1]{\markboth{\thechapter.\ #1}{}} 
\renewcommand{\sectionmark}[1]{\markright{\thesection\ #1}}
\fancyhf{}
\fancyhead[LE,RO]{\bfseries\thepage} 
\fancyhead[LO,RE]{\bfseries\leftmark}
\fancypagestyle{plain}{%
\fancyhead{} % leva l'intestazione
\renewcommand{\headrulewidth}{0pt} % e la linea
}

%%INDICE%%==============================================
\renewcommand{\contentsname}{Indice}
\renewcommand{\listfigurename}{Elenco delle Figure}
\renewcommand{\listtablename}{Elenco delle Tabelle}
\tableofcontents

\linespread{1.5}\selectfont		%interlinea
%%CAPITOLI===============================================
%%CAPITOLO 1: Introduzione=======================================
\chapter{Introduction}

\section{Purpose}

Our team will develop Travlendar+, a calendar-based application that aims to provide a schedule of user appointments, giving a plan to organize his daily life.
The main goals the app must fulfill are:

\begin{enumerate}
\renewcommand\labelenumi{\textbf{G\theenumi}}
%\item Schedule user appointments according to his necessities;% and his preferences, identifying the best mobility options; 
%\label{goal:G1}
%\item Make sure that the user can be in time for his appointments; \label{goal:G3}
%\item Optimize the schedule with respect to some criteria and constraints chosen by the user; \label{goal:G4}
%\item Provide a way to move between appointments location using several kinds of travel means; \label{goal:G5}
%\item Localize public travel means or sharing services, and buy tickets or book a ride, respectively; \label{goal:G6}
%\item Arrange the trips of the user, allowing him to locate travel services and buy public travel means tickets or book a sharing service; \label{goal:G7}
%\item Create a system with a \textbf{graphic} user interface, in order to simplify input/output interactions with the user. \label{goal:G8}
%\end{enumerate}

\item Allow the user to insert a list of appointments according to his necessities and his preferences;  \label{goal:G1}

\item The system S.P.W. to create a valid schedule of the user appointments (fare ref alla def) \label{goal:G2}

%comprare biglietti

\item 

\end{enumerate}


\section{Scope}
%The world and the machine is a way of thinking about problems and phenomena relevant in the reality of the software system we are to develop. The world contains the phenomena that occours in reality but are not observable by our system; the machine instead contains the phenomena which take place inside the system and are not observable from outside of it. The intersection between these two sets of phenomena is the so called set of shared phenomena, which contains the phenomena that are controlled by the world and observed by the machine or controlled by the machine and observed by the world.
Here we provide a brief description of the aspects of the reality of interest which the application is going to interact with.

User can receive an appointment on a certain date, time and location (over a region), that can be reached using different available travel means. The appointment can be held either at a specific time or in a time interval and lasts for a certain amount of time. An appointment can be recurrent, in other words, it repeats regularly over time (e.g., lunch, training, etc.). User can travel with someone else and can pick up or leave off these people during the day.

User can have his own travel means and a pass for public transportation. 
The travel means considered in this scenario can be grouped in three categories: public, shared or private.
\begin{itemize}
\item Public travel means: these include trains, buses, underground, taxis, trams. They have to be taken in their \textbf{appositi} stops. User must have a valid ticket in order to get on a public travel means (except for taxis, that pick up the user wherever he wants upon a call and do not require any ticket); 
\item Shared travel means: these include car and bike. They are located in specific places and require a reservation in order to be used by the user;
\item Private travel means: vehicles owned by the user. They can be cars, bikes, motorbikes.
\end{itemize}

Weather conditions can change during the day affecting usable travel means.
At the beginning of the day, or on demand, user can request a schedule of his daily appointments, following some criteria evaluated according to their assigned priority and satisfying some constraints imposed by the user.
When a new appointment is received, user creates a new item in the application and saves it in the appointment list. User can request a reschedule to the application due to unexpected changes of his plan (e.g. a cancelled appointment).

\subsection{World Phenomena}
\begin{itemize}
\item User receives a new appointment;
\item User picks up a person;
\item User owns private travel means and/or passes for public transportation;
\item User wakes up;
\item User pass expires.
\end{itemize}

\subsection{Shared Phenomena}
\begin{itemize}
\item Shared travel mean moves;
\item Shared travel mean its not available anymore;
\item Wheather condition changes;
\item Public travel means reach a stop-place;
\item Public travel means are late; 
\item Public travel means are not available due to a strike day;
\item User requests a schedule to the machine;
\item User inserts a new appointment into the application;
\item User requests to book rides;
\item User moves.
\end{itemize}

\section{Definitions, Acronyms, Abbreviations}

\newtheorem{mydef}{Definition}

%/ = sinonimo
sinonimi:
Appointment/meeting
Schedule/Scheduler
System/Application
preferences/constraint

%\begin{mydef} \label{definition: 1}
%Here is a new definition
%\end{mydef} 
%
%\ref{definition: 1}

def:
preferences: constraints on appointments or schedules
Opt Criteria: criteria followed by the scheduler in order to optimize

Schedule: a set of time-ordered and not overlapping appointments where their starting times are fixed and they're linked each other by a path travelled with a specific transportation mean

Valid Schedule: a Schedule which:
\begin{itemize}
\item is optimized according to the criteria chosen by the user;
\item ensures that the user will be on time for all his appointments;
\item respects the constraints imposed by the user
\end{itemize}


convenzioni:
variables are italic
states are bold

abbr:
GPS
GUI: graphic user interface
ETA: estimated time of arrival
Should provide a way: SPW
S.B.A.: should be able 
API

%\reqref{req:R1}
%\goalref{goal:G1}

\section{Revision history}

\section{Reference documents}

\section{Document structure}
\chapter{Architectural Design}

\section{Overview}

Travlendar+ has a multi-tier architecture.

\begin{figure}[H]
\begin{center}
\includegraphics[width=\textwidth, keepaspectratio]{"image/components".png}
\caption{High level components}
\end{center}
\end{figure}
 
User can interact directly with both version of the application: desktop and mobile, both laying on their respective OS APIs. The schedule computation is performed locally on the device in which the application runs by querying the GPS of the device and the external APIs:

\begin{itemize}
\item Mapping Service API
\item Travel Means APIs
\item Wheather Forecasts APIs
\end{itemize}

The communication between these components happens through different interfaces (wrappers), in order to improve the capability of the system to be expanded. In this way we provide a common pattern for the data acquired from the APIs, so that new ones can be easily included without the need of drastic changes on the central core.
In the architecture there are two more components: the Application server, that offers authentication and sinchronization services and the Database used to grant the durability of the data. At the end of the registration phase a new record is allocated in the Database and it will be used for check the identity of a user during the login phase. The Database also stores the user data (\ref{def:userdata}) so that they can be sinchronized in all the devices in which the user is logged in. 

\begin{figure}[H]
\begin{center}
\includegraphics[width=200pt, keepaspectratio]{"image/TierStack".png}
\caption{System stack}
\end{center}
\end{figure}

The figure above represents the stack architecture of our application and shows the hierarchy of the layers. Each level can communicate with the upper and lower layer.

\section{Component View}

\begin{figure}[H]
\begin{center}
\includegraphics[width=\textwidth, keepaspectratio, angle=90, origin=c]{"image/ComponentDiagram".png}
\caption{Component diagram}
\end{center}
\end{figure}


The application \footnote{in the image is reported only the mobile version of the application since the desktop version is identical to it, with the only difference that it can't exploit GPS services, so for mantaining the graphic clear only one of the two is reported.}
includes several components: 
\begin{itemize}

\item Identity Manager: it requests a service to the authentication provider either for the login or the registration of a user. 
%using the API offered by the OS it can communicate with the authentication manager that can write into the Database.

\item Appointment Manager: handles all the operations affecting appointment like modification, deletion and creation.
 
\item Scheduler: it computes and saves the schedules of the user.

\item Schedule Manager: handles all the operations affecting the schedule like showing and selecting the schedule to be run, notifying the user if a better schedule involving shared means has been found, buying the ticket for the means within the schedule. 

\item Runtime Schedule Manager: shows the directions and the progress of the running schedule.

\item Synchronization Manager: it requests a synchronization to the Synchronization provider when needed.

\item Mapping Service API Wrapper: manipulates data retrieved from a mapping service (e.g. Google Maps)in a format recognized by the application.

\item Travel Mean API Wrapper: manipulates data retrieved from a travel mean API service (e.g. Mobike, ATM, etc) in a format recognized by the application.

\item Weather Forecast API Wrapper: manipulates data retrieved from a weather forecast API service (e.g. Dark Sky) in a format recognized by the application.

\end{itemize}

The application Server is composed by the following two components:
\begin{itemize}

\item Authentication Provider: it handles the registration and the login phase of the user. 

\item Synchronization Provider: it has the purpose of synchronizing the user data (\ref{def:userdata}) in the database when something is changed.
 
\end{itemize}

The three components representing services (Wheather Forecast API, Travel Mean API, Mapping Service API) aren't included in the system because they are external services but they have been represented since our system is using the interfaces exposed from them.

\section{Deployment View}

\begin{figure}[H]
\begin{center}
\includegraphics[width=\textwidth, keepaspectratio]{"image/DeploymentDiagram".png}
\caption{Deployment Diagram}
\end{center}
\end{figure}

\section{Runtime View}

\begin{figure}[H]
\begin{center}
\includegraphics[width=\textwidth, keepaspectratio, angle=90,origin=c]{"image/RuntimeViews/Login".png}
\caption{Login runtime view}
\end{center}
\end{figure}

\begin{figure}[H]
\begin{center}
\includegraphics[width=\textwidth, keepaspectratio, angle=90,origin=c]{"image/RuntimeViews/AppointmentCreating".png}
\caption{Appointment creation runtime view}
\end{center}
\end{figure}

\begin{figure}[H]
\begin{center}
\includegraphics[width=\textwidth, keepaspectratio, angle=90,origin=c]{"image/RuntimeViews/AppointmentEditing".png}
\caption{Appointment editing runtime view}
\end{center}
\end{figure}

\begin{figure}[H]
\begin{center}
\includegraphics[width=\textwidth, keepaspectratio, angle=90,origin=c]{"image/RuntimeViews/DynamicDirections".png}
\caption{Dynamic directions runtime view}
\end{center}
\end{figure}

\begin{figure}[H]
\begin{center}
\includegraphics[width=\textwidth, keepaspectratio, angle=90,origin=c]{"image/RuntimeViews/Registration".png}
\caption{Registration runtime view}
\end{center}
\end{figure}

\begin{figure}[H]
\begin{center}
\includegraphics[width=710pt, keepaspectratio, angle=90,origin=c]{"image/RuntimeViews/ScheduleAppointments".png}
\caption{Schedule appointments runtime view}
\end{center}
\end{figure}

\begin{figure}[H]
\begin{center}

\includegraphics[width=\textwidth, keepaspectratio, angle=90,origin=c]{"image/RuntimeViews/RecoverCredentials".png}
\caption{Recover Credentials Runtime View}
\end{center}
\end{figure}

\begin{figure}[H]
\begin{center}
\includegraphics[width=\textwidth, keepaspectratio, angle=90,origin=c]{"image/RuntimeViews/BookingPhase".png}
\caption{Booking Phase Runtime View}
\end{center}
\end{figure}

\begin{figure}[H]
\begin{center}
\includegraphics[width=\textwidth, keepaspectratio, angle=90,origin=c]{"image/RuntimeViews/NotifySharedMeans".png}
\caption{Notify Shared Means Runtime View}
\end{center}
\end{figure}

%\begin{figure}[H]
%\begin{center}
%\includegraphics[width=720pt, keepaspectratio, angle=90,origin=c]{"image/RuntimeViews/ScheduleAppointments".png}
%\caption{Schedule Appointments Runtime View}
%\end{center}
%\end{figure} èDOPPIA

\begin{figure}[H]
\begin{center}
\includegraphics[width=720pt, keepaspectratio, angle=90,origin=c]{image/RuntimeViews/ScheduleSelection".png}
\caption{Schedule selection runtime view}
\end{center}
\end{figure}

\subsection{Note on Runtime Views}
In tbe diagrams the interaction involving Application Server, Application and API wrappers since they are not informative to show the behaviour of the application.

above only the interactions between internal components are shown.

\section{Component Interfaces}
\begin{figure}[H]
\begin{center}
\includegraphics[height=230pt, keepaspectratio, angle=90,origin=c]{image/InterfacesDiagrams".png}
\caption{Schedule selection runtime view}
\end{center}
\end{figure}

\section{ER Schema}
\begin{figure}[H]
\begin{center}
\includegraphics[width=580pt, keepaspectratio, angle=90,origin=c]{image/SchemaER/ER_DD.jpeg}
\caption{application database ER schema}
\end{center}
\end{figure}

\section{Selected architectural styles and patterns}

\begin{itemize}
\item MVC: all the classes of the application will adopt this pattern
\item Adapter: API wrappers will adopt this pattern
\item Client-Server: the application  running on a device represents the client part, reqesting services to the Travlendar server
\item OAuth: standard adopted by the exposed APIs of the server, in order to get authorization to make request from different clients
\item Restful API: a particular kind of APIs that encapsulate the request/response body in a JSON string
\item Singleton: restricts the instantiation of a class to one object. This is useful when exactly one object is needed to coordinate actions across the system. Some classes of our system can be built on this pattern, like the Travel Mean.
\end{itemize}


%\section{Other Design Decisions}
%
%\subsection{API Wrappers Class Diagrams}
%
%
%
%\begin{figure}[H]
%\begin{center}
%\includegraphics[width=\textwidth, keepaspectratio ,origin=c]{"image/WrapperUMLDiagram".png}
%\caption{API wrapper class diagram}
%\end{center}
%\end{figure}
\chapter{Algorithmic Design}
In this chapter are given the guidelines on how to implement the most important functionalities that the components of the system will offer. The pseudocode of the relevant method is shown. 

\section{Login}
The Login involves the application the application server and the database. The last one is located outside the application server. 
To login in to the application the following steps must be done:
\begin{enumerate}
\item the user presses the Login button in the Login view;
\item the Identity manager (Login method) sends a request to the Application Server for authentication
\item A query on the database is performed to check the validity of the user credential and a response to the client is returned;
\end{enumerate}

\begin{lstlisting}
//client side
Login(username, password)
	response = SendRequest("api/login", username, password)
	if response.isValid 
		token = response.GetToken
		Show(HomeView)
	else
		Show(ErrorMessage , "Invalid Credentials")

//server side
LoginRequest(username, password)
	result = database.query(username, password)
	if result == 1
		SendAuthenticationResponse (token)
	else
		sendAuthenticationError(error code)
\end{lstlisting}

\section{CreateSchedule}
The process of computing a schedule is composed by the following steps:

\begin{enumerate}
\item Create the predecessor matrix (P) in which every cell (i,j) contains:

\[
    P_{ij}=
    \begin{cases}
      1, &\text{if $Appointment_i$ must precede $Appointment_j$} \\
      0, &\text{if $Appointment_i$ must follows $Appointment_j$} \\
      -1, &\text{if $Appointment_i$ can be scheduled both before or after $appointment_j$}
    \end{cases}
\]

\item Compute all the possible ordered arrangements of appointments with respect to the predecessor matrix
\item For each appointment in each arrangement, set the starting time according to travel time. This is estimated considering the euclidean distance and heuristics on the kind of travel mean between every pair of consecutive appointments in the arrangement, considering the bestTravelMean; \ref{def:bestTravelMean}
\item Check the feasibility of the arrangements and discard the ones which have overlapping appointments;
\item Choose the most convenient one, according to the optimization criteria.
\item Call mapping service Api to fix the effective routes between the appointments
\end{enumerate}

\begin{lstlisting}
ComputeSchedule(wakeUpTime, startingLocation, appts, constraints, optCriteria)
	p=CalculatePredecessorMatrix(appts)
	a=CalculateArrangements(appts, p, 0, 1)
	SetStartingTime(a, startingTime, startingLocation, constraint, optCriteria)
	s=ChooseBestSchedule(a)
	MappingServiceRequest(s)
	return s

CalculatePredecessorMatrix(appts)
	p=new Matrix[appt.size,appt.size]
	for(i=0 .. appts.size)
		for(j=i+1 .. appts.size)
			a1=appts[i]
			a2=appts[j]
			if a1.deterministic && a2.deterministic
				if a1.startingTime < a2.startingTime
					pred[i,j]=1
				else
					pred[i,j]=0
			elseif a1.deterministic && !a2.deterministic
				if a1.startingTime < a2.timeSlot.start
					pred[i,j]=1
				elseif a1.endingTime > a2.timeSlot.end
					pred[i,j]=0
				else
					pred[i,j]=-1
			elseif !a1.deterministic && a2.deterministic
					if a1.timeslot.end < a2.endingTime
						pred[i,j]=1
					elseif a1.timeSlot.start > a2.startingTime
						pred[i,j]=0
					else
						pred[i,j]=-1
			elseif !a1.deterministic && !a2.deterministic
					if a1.timeSlot.end < a2.timeSlot.start
						pred[i,j]=1
					elseif a1.timeSlot.start > a2.timeSlot.end
						pred[i,j]=0
					else
						pred[i,j]=-1
	return p
		
		
CalculateArrangements(appts, p, curri, currj)
	arrangement=new List	
	for(i=curri .. appts.size-1)
		for(j=currj-1 .. appts.size)
			if p[i,j]==-1
				p0=p
				p0[i,j]=0
				CalculateArrangements(appts, p0, i, j)
				
				p1=p
				p1[i,j]=1
				CalculateArrangements(appts, p1, i, j)
				
				return
				
	a=ConvertPredMatrixToList(appts,p)
	arrangement.addLast(a)
	return arrangement

		 
ConvertPredMatrixToList(appts, p)
//converts a "-1 free" predecessor matrix to an ordered list of appointments


SetStartingTime(a, startingTime, startingLocation, constraint, optCriteria)
	for(arr in a)
		dummyStartingAppt = new appointment(startingTime, startingLocation, duration=0)
		arr.addFirst(dummyStartingAppt)
		for(i=1 .. arr.size)
			appt1=arr[i-1]
			appt2=arr[i]
			travelMean=getBestTravelMean(appt1, appt2, constraint, optCriteria)
			travelTime=travelMean.estimateTime(appt1, appt2)
			if appt2.deterministic 
				appt2.startingTravelTime = appt2.startingTime-travelTime
				appt2.travelMean=travelMean
				if appt1.endingTime > appt2.startingTravelTime
					error("schedule not feasible")
			else
				appt2.startingTravelTime = max(appt1.endingTime,appt2.timeSlot.start-travelTime)
				appt2.travelMean=travelMean
				if appt2.startingTravelTime > appt2.timeSlot.end
					error("schedule not feasible")
					
					
getBestTravelMean(appt1, appt2, constraint, optCriteria)
	l=getNotConstrainedTravelMeans(constraint)
	for(t in l)
		switch optCriteria
			case "MoneySpent"
				t.cost=estimateMoney(t, appt1, appt2)
			case "Time"
				t.cost=estimateTime(t, appt1, appt2)
			case "CarbonFootprint"
				t.cost=estimateCarbon(t, appt1, appt2)
	sortByCriteria(l, optCriteria)
	return l
	
ChooseBestSchedule(a)
	best=a[0]
	for(i = 0 .. a.size)
		sum=0
		for(appt in a[i])
			sum+=appt.travelMean.cost
		a[i].totalCost=sum
		if sum<a[0].totalCost
			best=a[i]
	return best
	
MappingServiceRequest(s)
	for (i=0 .. s.size-1)
		response=MappingServiceRequest(s[i], s[i+1])
		s[i].path=response.path
		s[i].startingTime=response.startingTime

\end{lstlisting}

\section{Registration}
The Registration involves the application the application server and the database. The last one is located outside the application server. 
To register in to the application the following steps must be done:
\begin{enumerate}
\item The user presses the Registration button in the Login view;
\item The Identity manager (Registration method) sends a request to the Application Server for authentication
\item A query on the database is performed to check the presence of the user credential and a confirmation email is sent
\item The user confirms the email by clicking on the designated link
\item The user's state on the database becomes confirmed
\end{enumerate}

\begin{lstlisting}
//client side
Register(username, password)
	response = SendRequest("api/registration", username, password)
	if !response.valid
		Show(ErrorMessage , "User Already Registered")
		

	
//server side
RegistrationRequest(username, password)
	result = database.query(username, password)
	if result == 0
		database.insertTuple(username, password)
		sendEmail(username)
	else
		sendAuthenticationError(error code)
		
EmailConfirmationRequest(username)
	result = database.modifyTuple(username, confirmed=true)
\end{lstlisting}


\section{Synchronize}
The synchronization in our system is the process that aims to keep the data consistent and updated between different devices. For example, when a user inserts an appointment on one of his devices, then the changes must be propagated to all his other devices, once the login is performed. 
The Synchronization involves the application, the application server and the database. The last one is located outside the application server. To sinchronize data across multiple devices two actions must be carried out:
\begin{itemize}
\item Upload of local data on the database when a single change on the  client's local data occours(Synchronize Upwards);
\item Download of data from the database, if an update is necessary, when login is performed(Synchronize Downwards).
\end{itemize}

\begin{lstlisting}
//client background processes

SynchronizeUpwards(changedData)
	update = false
	while !update
		response = SendRequest("api/sync/up", token, changedData)
		if response	 
			update = true
			
SynchronizeDownwards()
	newData=SendRequest("api/sync/down", token)
	localData=newData
		
		
//server side

SynchUpwardsRequest(token, changedData)
	userID = database.getUser(token)
	if changedData.action == insert
		//changedData is a newly inserted element
		database.insert(changedData, userID)
	else if changedData.action == edit
		database.update(changedData, userID)
	else
		database.delete(changedData, userID)
	sendResponse("syncresult", true)
	
SynchUpwardsRequest(token, changedData)
	userID = database.getUser(token)
	data=database.getUserData(userID)
	sendResponse(data)

\end{lstlisting}




		
\chapter{User Interface Design}
We refer to chapter 3.1.1 of the RASD, where we have given a complete overview of how the user interfaces will look like.
\chapter{Requirement Traceability}
In each section section we will describe how the requirements of each goal are mapped in the design elements described above.

\section{Goal 1}

\begin{itemize}

\item \textit{The system S.B.A to handle a registration phase in 
which the user will provide an e-mail and a password}

\begin{itemize}
\item Login Interface
\item Identity Manager
\item Authentication Provider
\item Database
\end{itemize}

\item \textit{The system S.B.A. to verify the e-mail given by the user by sending a confirmation e-mail to his address}

\begin{itemize}
\item Login Interface
\item Identity Manager
\item Authentication Provider
\item Database
\end{itemize}

\item \textit{The system should let the user to specify his parameters}

\begin{itemize}
\item User Account Interface
\end{itemize}

\end{itemize} 


\section{Goal 2}

\begin{itemize}

\item \textit{The system S.B.A to recognize a registered user given an e-mail and a password}

\begin{itemize}
\item Login Interface
\item Identity Manager
\item Authentication Provider
\item Database
\end{itemize}

\item \textit{The system S.B.A. to retrieve information from the user about his registration informations, i.e. his e-mail and password}

\begin{itemize}
\item Login Interface
\item Identity Manager
\item Authentication Provider
\item Database
\end{itemize}

\item The system must update the local date if some changes on them have been done from other devices.

\begin{itemize}
\item Synchronization Manager
\item Synchronization Provider
\item Database
\end{itemize}

\end{itemize}

\section{Goal 3}

\begin{itemize}

\item \textit{The system should be able to retrieve an e-mail address from the user}

\begin{itemize}
\item Login Interface
\end{itemize}

\item \textit{ The system should be able to send the password of a user to his e-mail given it}

\begin{itemize}
\item Authentication Provider
\item Database
\end{itemize}

\end{itemize}


\section{Goal 4}

\begin{itemize}

\item \textit{The system S.B.A. to retrieve information from the user about his appointments}

\begin{itemize}
\item Appointment CRUD interface.  
\end{itemize}

\item \textit{The system S.B.A. to store an appointment in his memory}

\begin{itemize}
\item Appointment Manager
\end{itemize}

\end{itemize}


\section{Goal 5}

\begin{itemize}

\item \textit{The system should let the user change the parameters and the constraints of an inserted appointment}

\begin{itemize}
\item Appointment CRUD interface.
\item Appointment Manager
\end{itemize}

\item \textit{The system S.B.A to rewrite the appointment in his memory with his new parameters}

\begin{itemize}
\item Appointment Manager
\end{itemize}

\end{itemize}


\section{Goal 6}

\begin{itemize}

\item \textit{ Allow the user to set the constraints of the schedule}
\begin{itemize}
\item Schedule Interface
\end{itemize}

\item \textit{Allow the user to set the optimization criteria for the schedule}
\begin{itemize}
\item Schedule Interface
\end{itemize}

\item \textit{Allow the user to set the variables for the schedule}
\begin{itemize}
\item Schedule Interface
\end{itemize}

\item \textit{The system S.B.A. to gather information from external APIs about: 
	\begin{itemize}
		\item travel options with related travel option data; 
		\item weather forecast;
		\item strike days;
		\item delays.
	\end{itemize}}

\begin{itemize}
\item Mapping Service API Wrapper
\item Travel Mean API Wrapper
\item Weather Forecast API Wrapper
\end{itemize}

\item \textit{The system S.B.A. to select the best travel option according to the optimization criteria taking into account:
	\begin{itemize}
		\item user constraint;
		\item user parameters;
		\item travel option data; 
		\item weather forecast;
		\item information about strike day;
		\item information about delays.
	\end{itemize}}

\begin{itemize}
\item Scheduler
\end{itemize}

\item \textit{ The system S.B.A to store valid schedules requested by the user}
\begin{itemize}
\item Schedule Manager
\end{itemize}

\end{itemize}


\section{Goal 7}

\begin{itemize}

\item \textit{The system should let the user accept a schedule from the saved ones}
\begin{itemize}
\item Schedule List Interface
\item Schedule Manager
\end{itemize}

\end{itemize}

\section{Goal 8}

\begin{itemize}

\item \textit{The system S.B.A to book a travel mean through external API offered by third part application in which the user is signed}
\begin{itemize}
\item Schedule Manager
\item Travel Mean API Wrapper
\end{itemize}

\end{itemize}


\section{Goal 9}

\begin{itemize}

\item \textit{The system S.B.A to retrive the position of the user from his GPS}

\begin{itemize}
\item GPSAPI
\item Runtime Schedule Manager
\end{itemize}

\item \textit{ The system S.B.A. to retrive from an external API the directions to give to the user for reach the next appointment}

\begin{itemize}
\item Mapping Service API Wrapper
\item Runtime Schedule Manager
\end{itemize}

\item \textit{The system S.B.A. to retrive from an external API the Graphical representation of the path that will be travelled by the user}

\begin{itemize}
\item Mapping Service API Wrapper
\item Runtime Schedule Manager
\end{itemize}

\end{itemize}

\section{Goal 10}

\begin{itemize}

\item \textit{The system should be able to send notifications to the user}
\begin{itemize}
\item Schedule Manager
\end{itemize}

\item \textit{The system should be able to retrieve information about the availablity of shared travel means from external APIs without the user request}
\begin{itemize}
\item Mapping Service API Wrapper
\item Schedule Manager
\end{itemize}

\end{itemize}



\chapter{Implementation Integration and TestPlan}
In the next sections we will show the timeline of the implementation integration and testplan. This timeline is led by the functionalities, for each one we first implement the components needed, and then we proceed with the integration and with the test of the functionality.

\section{User Authentication}
Implemented and integrated Components:
\begin{itemize}
\item Authentication Provider
\item Database
\item Identity Manager
\end{itemize}
Functionality tested:
\begin{itemize}
\item User Registration
\item User Login
\item Recover credentials
\item SetUserParameters
\item GetUserParameters
\end{itemize}

\section{Appointments management}
Implemented and integrated components:
\begin{itemize}
\item Appointment manager
\end{itemize}
Functionality tested:
\begin{itemize}
\item Appointment creation
\item Appointment editing
\item Appointment deletion
\item Appointments retrieving
\end{itemize}

\section{External Api Data Retrieving}
Implemented and integrated components:
\begin{itemize}
\item Mapping Service API Wrapper
\item Travel Mean API Wrapper
\item Weather Forecast API Wrapper
\end{itemize}
Functionality tested:
\begin{itemize}
\item Weather forecast data retrieving
\item Strike day information retrieving
\item Travel option data retrieving
\item Get ticket price
\item Purchase ticket
\end{itemize}

\section{Static Schedule Management}
Implemented and integrated components:
\begin{itemize}
\item Scheduler
\item Schedule Manager
\end{itemize}
Functionality tested:
\begin{itemize}
\item Compute a schedule
\item Schedules retrieving
\item Schedule selection
\item Schedule editing
\item Schedule deletion
\item Schedule price updating
\end{itemize}

\section{Dynamic Schedule Management}
Implemented and integrated components:
\begin{itemize}
\item Runtime Schedule Manager
\end{itemize}
Functionality tested:
\begin{itemize}
\item Directions retrieving
\item Dynamic directions displaying
\end{itemize}

\section{Synchronization}
Implemented and integrated components:
\begin{itemize}
\item Synchronization provider
\item Synchronization Manager
\end{itemize}
Functionality tested:
\begin{itemize}
\item Synchronize
\end{itemize}


\chapter{Effort Spent}
\begin{itemize}
\item Federico Parroni: \textbf{44 hours};
\item Edoardo D'Amico: \textbf{44 hours};
\item Giovanni Gabbolini: \textbf{44 hours}.
\end{itemize}




%%APPENDICI==============================================
\appendix
%%\input{appendiceA}

\end{document}