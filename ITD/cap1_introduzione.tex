%%CAPITOLO 1: Introduzione=======================================
\chapter{Introduction}
After the release of the DD document the application Travlendar+ has been made up following the guidelines explained in the previous two documents. The application is a prototype, these means that not all the functionalities described in the RASD and DD documents have been implemented but only the most important ones. By the way the application manages to cover the basic needs of a user. Tests and debug of the application have been carried out on a virtual android device and on a real android device. For the implementation of this prototype a month and half has been spent by our three-people team.

\section{Front page}

\section{Purpose}

\section{Scope}
In this paper there is an overview of the main topics concerning how the implementation of the prototype of Travlendar+ application has been made. First of all the functionalities that are currently implemented in the software will be presented. then the adopted development frameworks  will be explained with its pros and cons. then it comes to the structure of the source code of the application. in the end the tests performed to test the right functioning of the application and their outcomes and how the various part of the application has been put together will be presented. In the last part of the document there is a brief guide on how to install the developed application.

\section{Revision history}