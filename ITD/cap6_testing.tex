\chapter{Testing}



\section{Server tests}

\subsection{Goal 1}
\testTable{Registration with empty email}{Incorrect email error response}{Incorrect email error response}
\testImage{"testServer/emptyEmail".png}

\testTable{Registration with empty password}{Incorrect password error response}{Incorrect email error response}
\testImage{"testServer/emptyPassword".png}

\subsection{Goal 2}
\testTable{Login with wrong client credentials}{Wrong credentials error response}{Wrong credentials error response}
\testImage{"testServer/wrongCredentials".png}

\testTable{Login with wrong user credentials}{Wrong username or password error response}{Wrong username or password response}
\testImage{"testServer/wrongUserPassword".png}

\subsection{Other tests}
\testTable{Access auth-protected api with correct access token}{Response succesfully returns}{Response succesfully returns}
\testImage{"testServer/correctToken".png}

\testTable{Access auth-protected api with wrong access token}{Wrong access token error response}{Wrong access token error response}
\testImage{"testServer/wrongToken".png}


\section{Client test}

\subsection{Goal 1}
\testTable{Registration with invalid email}{Incorrect email error message}{Incorrect email error message}
\testImage{"testClient/invalidRegistration".png}

\testTable{Registration with non matching passwords confirmation}{Non matching passwords error message}{Non matching passwords error message}
\testImage{"testClient/nonmatchingPassword".png}

\testTable{Registration with correct user credentials and password confirmation}{Registration succesfully done}{Registration done message}
\testImage{"testClient/correctRegistration".png}

\subsection{Goal 2}
\testTable{Login with wrong user credentials}{Wrong username or password error message}{Wrong username or password error message}
\testImage{"testClient/invalidLogin".png}

\subsection{Goal 4}
\testTable{The appointment list is blank and the user likes to insert a new appointment, say a Software Engineering II lesson dated 15-01-2018, located in Via Camillo Golgi 42 and which lasts 2 hours. So, from the appointments section, the users adds a new appointment with these peculiarities.}
{The appointment should be created and thus appear in the appointments list.}{As expected}

\testTable{The user wants to see the characteristics of a previously created appointment, say the Software Engineering II lesson created before, so the user taps the appointment on the appointments list}{A view should appear, showing the peculiarities of the selected appointment}{As expected}

\subsection{Goal 5}
\testTable{The user wants to modify the Software Engineering II lesson's details, setting the starting time at 10.30, so he taps the appointments and he clicks on the edit button and he modifies the starting time}{The appointment should be modified according to the new value of starting time imposed}{As expected}

\testTable{The user wants to delete the Software Engineering II appointment, so he holds the click to this appointment on the appointment view and decide to delete it}{The appointment should disappear from the view}{As expected}

\subsection{Goal 6}

The system S.P.W. to create a valid schedule (1.3.6) of the user appointments when
requested and display the scheduling result (1.3.9);

\TestTable{
user has no computed schedule in his schedule list.}
{Creation of a new schedule.}
{{\begin{enumerate}
\item a click on the add schedule button is performed and the user is redirected to the schedule creation view.
\item the user selects the date in which he/she wants compute his/her schedule.
\item the empty fields are filled by the user.
\item the button for computing the schedule is clicked and the progress bar is shown.
\item the user is redirected to the schedule list and there, is added the new computed schedule.
\item with one click on the created schedule the schedule results are shown.
\end{enumerate}}}
{The outcome is equal to the expected behaviour.}

\TestImage{test_goal_6/2}{execution of point 2}
\TestImage{test_goal_6/3}{execution of point 5}
\TestImage{test_goal_6/4}{execution of point 6}







